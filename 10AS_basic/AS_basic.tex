%----------------------------------------------------------------------------------------
%	PART
%----------------------------------------------------------------------------------------
\part{준비}

\usechapterimagetrue
\chapterimage{chapter_head_1.pdf} % Chapter heading image
\chapter{대기 과학 기초}\index{대기 과학 기초}


\section{과학 측정}\index{과학 측정}

\subsection{목적}\index{목적}

과학자들이 과학기기들을 사용하는 이유, 차원과 측정 단위, 과학표기법, 백분율 오차에 대해 알아보고자 한다.

\subsection{관측과 측정}\index{관측과 측정}

과학자들의 임무 중의 많은 부분은 자연을 관측(ovservation)\을 수행하는 것이 포함된다. 관측은 인간의 오감 중의 하나를 사용하여 간단히 이루어질 수 있다. 오감은 보고, 만지고, 냄새를 맡고, 맛을 보고, 듣는 것인데, 오늘날 과학자들은 과학기기(scientific instrument)들을 사용하여 그들 자신의 감각들을 확장한다. 
관측 기기들은 인간의 감각들을 확장하도록 고안되었고 자연 세계를 정확하게 측정할 수 있어야 한다. 과학 기기들에는 간단한 자(scale)에서부터 복잡한 위성 또는 전파 망원경까지 존재한다. 

측정(measurement)은 관찰을 보다 정확하게 이루어지도록 하며, 오늘날 과학자들은 자연의 물리적인 또는 화학적인 성질들을 확인하기 위하여 다음과 같은 네가지 기본적인 측정을 사용한다.
\begin{itemize}
	\item 길이(length) : 2개의 고정점들 사이의 거리
	\item 질량(mass) : 물체 내의 물질의 양을 재는 측정치, 지구상에는 물체의 무게(weight)를 결정하기 위해 측정
	\item 시간(time) : 전진적인 이동, 즉 2가지 사건들 사이의 임의의 간격
	\item 에너지(energy) :물질의 전하로서 측정되거나, 운동에너지로 불리는 물질의 이동의 측정치, 물질의 열, 온도가 포함됨.
\end{itemize}

\subsection{차원}\index{차원}
대기의 운동을 지배하는 기본 법칙들은 동일한 차원(demension)으로 이루어진다. 즉 기본 법칙을 나타내는 방정식의 모든 항들은 동일한 물리 차원을 가져야 한다. 기본 차원은 길이[$rmL$], 질량[$rmM$], 시간[$rmT$], 열역학적 온도[$rmK$]로 구성된다. 다른 차원을 이러한 기본 차원들을 서로 곱하고 나눈 항으로 표현할 수 있다. 예를 들면 속도의 차원[$rmLT^-1$]은 길이 차원[$rmL$]을 시간 차원[$rmT$]으로 나누면 된다. 운동 법칙을 표현하는 항들의 규모를 측정하고 비교하기 위하여 기본 차원에 대해서 측정 단위를 정의해야 한다.


\section{국제 단위계}\index{국제 단위계}
미터법에 따른 측정 단위를 국제적으로 통일한 체계로서 SI단위라고도 한다. 기본 단위로서 길이에 미터(m), 무게에 킬로그램(kg), 시간에 초(s), 전류에 암페어(A), 온도에 켈빈(K), 물질량에 몰(mol), 광도에 칸델라(cd)의 7가지 기본단위와 이로부터 유도된 유도단위가 있다. 1960년 제11회 국제도량형총회에서 결정하였다.
%[네이버 지식백과] 국제단위계 [International System of Units] (두산백과)

\subsection{기본 단위}\index{기본 단위}

\subsubsection{길이의 단위(미터)}\index{질량의 단위(킬로그램)}
백금-이리듐의 국제원기에 기초를 둔 미터의 정의는 제11차 국제도량형총회(1960)에서 크립톤 86 원자(86Kr)의 복사선 파장에 근거를 둔 정의로 대체되었다. 이 정의는 미터 현시의 정확도를 향상시키기 위하여 채택되었다. 이 정의는 1983년의 제17차 국제도량형총회에서 다시 다음과 같이 대체되었다. 미터는 빛이 진공에서 1/299792458 초 동안 진행한 경로의 길이이며, 그 기호는 “m"\로 한다.

\subsubsection{질량의 단위(킬로그램)}\index{질량의 단위(킬로그램)}
백금-이리듐으로 만들어진 국제원기는 1889년 제1차 국제도량형총회에서 지정한 상태 하에 국제도량형국(BIPM)에 보관되어 있으며 당시 국제도량형총회는 국제원기를 인가하고 다음과 같이 선언하였다. 이제부터는 이 원기를 질량의 단위로 삼는다. 킬로그램은 질량의 단위이며 국제킬로그램원기의 질량과 같으며, 그 기호는 “㎏”으로 한다.

\subsubsection{시간의 단위 (초)}\index{시간의 단위 (초)}
예전에는 시간의 단위인 초를 평균 태양일의 1/86400로 정의하였었다. 그러나, 지구 자전주기의 불규칙성 때문에 이 정의를 우리가 요구하는 정확도로 실현할 수 없다는 것이 측정에 의해 밝혀졌다. 이후 시간의 단위를 원자나 분자의 두 에너지 준위 사이의 전이에 기초를 둔 원자시간표준이 실현가능하고 훨씬 더 정밀하게 재현될 수 있다는 것이 실험에 의해 증명됨에 따라 1968년 제13차 국제도량형총회에서 초의 정의를 다음과 같이 바꾸었다. 초는 세슘-133원자(133Cs)의 바닥상태에 있는 두 초미세 준위 사이의 전이에 대응하는 복사선의 9192631770 주기의 지속시간이며, 그 기호는 “s"로 한다.

\subsubsection{전류의 단위 (암페어)}\index{전류의 단위(암페어)}
전류와 저항에 대한 소위 "국제" 전기단위는 1893년 국제전기협의회에서 최초로 도입되었고, 1948년 제9차 국제도량형총회에서 전류의 단위인 암페어를 다음과 같이 정의하였다.
암페어는 무한히 길고 무시할 수 있을 만큼 작은 원형 단면적을 가진 두 개의 평행한 직선 도체가 진공 중에서 1 미터의 간격으로 유지될 때, 두 도체 사이에 매 미터 당 2x10-7 뉴턴(N)의 힘을 생기게 하는 일정한 전류이다.

\subsubsection{열역학적 온도의 단위 (켈빈)}\index{열역학적 온도의 단위 (켈빈)}
열역학적 온도의 단위는 실질적으로 1954년 제10차 국제도량형총회에서 정해졌는데, 여기서 물의 삼중점을 기본 고정점으로 선정하고 이 고정점의 온도를 정의에 의해서 273.16 K로 정했다. 이후 1968년 제13차 국제도량형총회에서 “켈빈도”(기호 °K) 대신 켈빈(기호 K)이라는 명칭을 사용하기로 채택하였고, 열역학적 온도의 단위를 아래와 같이 정의하였다. 켈빈은 물의 삼중점에 해당하는 열역학적 온도의 1/273.16 이며, 그 기호는 "K"로 한다.

\subsubsection{물질량의 단위 (몰)}\index{물질량의 단위 (몰)}
국제순수응용물리학연맹, 국제순수응용화학연맹, ISO의 제안에 따라 국제도량형총회에서는 1971년에 “물질량”이란 양의 단위의 명칭은 몰(기호 mol)로 정하고 몰의 정의를 다음과 같이 채택하였다.
몰은 탄소 12의 0.012 킬로그램에 있는 원자의 개수와 같은 수의 구성요소를 포함한 어떤 계의 물질량이다. 그 기호는 "mol"이다.

\subsubsection{광도의 단위 (칸델라)}\index{광도의 단위 (칸델라)}
1948년 이전에는 광도의 단위를 불꽃이나 백열필라멘트 표준에 기초를 두고 사용하였으나 이후 백금응고점에 유지된 플랑크복사체(흑체)의 광휘도에 기초를 둔 “신촉광(新燭光)”으로 대치되었었다. 그러나 고온에서 플랑크 복사체를 현시하기에 어려움이 많아 1979년 제16차 국제도량형총회에서 다음과 같은 새로운 정의를 채택하였다.
칸델라는 진동수 540×1012 헤르츠인 단색광을 방출하는 광원의 복사도가 어떤 주어진 방향으로 매 스테라디안(sr) 당 1/683 와트일 때 이 방향에 대한 광도이다.

\begin{table}[h]
	\centering
	\caption{기본 단위}
\begin{tabular}{c|c|c}
	\hline 
특성	& 이름 & 기호 \\ 	\hline 
길이	& 미터 & m \\ 	\hline 
질량	& 킬로그램 & kg \\  \hline 
시간	& 초 &  s \\ 	\hline 
열역학적 온도	& 켈빈 & K \\  	\hline 
전류	& 암페어 & A \\  	\hline 
물질량	& 몰  & mol \\  	\hline 
광도	& 칸델라 & cd \\ 	\hline 
\end{tabular} 
		\label{table:unit01}
\end{table}

%[네이버 지식백과] 국제단위계 [International System of Units] (두산백과)

\subsection{유도 단위}\index{유도 단위}
유도 단위는 기본 단위의 곱 또는 나눔으로 만들어진 단위이다. 유도 단위의 표현에는 기본 단위 외의 다른 인자가 나타나지 않기 때문에 SI 단위가 일관성을 갖게 되고, 또한 계산할 때 다른 환산인자를 필요로 하지 않는다. 이 유도 뒨위 중 21개에는 편의상 특별한 명칭과 기호가 주어져 있다(\ref{table:unit02}). 

\begin{table}[h]
	\centering
	\caption{특별한 이름을 갖는 유도 단위}
	\begin{tabular}{c|c|c}
		\hline 
특성	& 이름 & 기호 \\ 	\hline 
평면각	& 라디안(radian) & rad  \\ 	\hline 
입체각	& 스테라디안(steradian) & $\rm sr$ \\  \hline 
진동수, 주파수 & 헤르츠(hertz) & $\rm Hz(s^{-1})$  \\  \hline 
힘 & 뉴턴(newton) & $\rm N~(kg ~ m ~ s^{-1})$ \\  \hline 
압력, 응력 & 파스칼(pascal) & $\rm Pa~(N ~ m^{-2})$  \\  \hline 
에너지, 일, 열량 & 줄(joule) &   \\  \hline 
일률, 전력 & 와트 &  \\  \hline 
전하량 & 쿨룸 &  \\  \hline 
전위, 전압, 기전력 & 볼트 &  \\  \hline 
전기용량 & 패럿 &  \\  \hline 
전기저항& 옴 &  \\  \hline 
전기전도도 & 지멘스 &  \\  \hline 
자기력 선속 & 웨버&  \\  \hline 
인덕턴스 & 헨리 &  \\  \hline 
섭씨 온도 & 섭씨도 &  \\  \hline 
광선속& 루멘 &  \\  \hline 
조명도& 럭스 &  \\  \hline 
방사능 & 베크렐 &  \\  \hline 
습수선량, 비에너지투여 & 그레이 &  \\  \hline 
선량당량, 선당량지수 & 시버트 &  \\  \hline 
	\end{tabular} 
	\label{table:unit02}
\end{table}
