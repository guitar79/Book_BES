\part{파이썬 (Python)}

%----------------------------------------------------------------------------------------
%	CHAPTER 
%----------------------------------------------------------------------------------------

\chapterimage{chapter_head_1.pdf} % Chapter heading image

\chapter{Basic Phthon}

\section{Python 소개}\index{Python 소개}

파이썬(Python)은 1991년 프로그래머인 귀도 반 로섬(Guido van Rossum)이 발표한 고급 프로그래밍 언어로, 플랫폼 독립적이며 인터프리터식, 객체지향적, 동적 타이핑(dynamically typed) 대화형 언어이다. 파이썬이라는 이름은 귀도가 좋아하는 코미디 〈Monty Python's Flying Circus〉에서 따온 것이다.

파이썬은 비영리의 파이썬 소프트웨어 재단이 관리하는 개방형, 공동체 기반 개발 모델을 가지고 있다. C언어로 구현된 C파이썬 구현이 사실상의 표준이다.

파이썬은 무료이며 누구나 다운받아 사용 가능 하다.

\section{Python 기초}\index{Python 기초}

파이썬(Python)은 1991년 프로그래머인 귀도 반 로섬(Guido van Rossum)이 발표한 고급 프로그래밍 언어로, 플랫폼 독립적이며 인터프리터식, 객체지향적, 동적 타이핑(dynamically typed) 대화형 언어이다. 파이썬이라는 이름은 귀도가 좋아하는 코미디 〈Monty Python's Flying Circus〉에서 따온 것이다.

파이썬은 비영리의 파이썬 소프트웨어 재단이 관리하는 개방형, 공동체 기반 개발 모델을 가지고 있다. C언어로 구현된 C파이썬 구현이 사실상의 표준이다.

파이썬은 무료이며 누구나 다운받아 사용 가능 하다.


\section{Python 설치}\index{Python 설치}

파이썬(Python)은 1991년 프로그래머인 귀도 반 로섬(Guido van Rossum)이 발표한 고급 프로그래밍 언어로, 플랫폼 독립적이며 인터프리터식, 객체지향적, 동적 타이핑(dynamically typed) 대화형 언어이다. 파이썬이라는 이름은 귀도가 좋아하는 코미디 〈Monty Python's Flying Circus〉에서 따온 것이다.

파이썬은 비영리의 파이썬 소프트웨어 재단이 관리하는 개방형, 공동체 기반 개발 모델을 가지고 있다. C언어로 구현된 C파이썬 구현이 사실상의 표준이다.

파이썬은 무료이며 누구나 다운받아 사용 가능 하다.

