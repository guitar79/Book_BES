%----------------------------------------------------------------------------------------
%	PART
%----------------------------------------------------------------------------------------
\part{기초지구과학I}

\usechapterimagetrue

\chapterimage{chapter_head_1.pdf} % Chapter heading image
\chapter{우리의 지구}\index{우리의 지구}

\section{지구의 모양과 크기}\index{지구의 모양과 크기}

\subsection{지구의 모양}\index{지구의 모양}

\subsubsection{지구타원체}\index{지구타원체}

지구가 둥글다는 사실은 이미 1세기경에 일반적인 사실로 받아들여졌다. 중세 시대에 지구가 평평하다는 주장이 다시 제기되었으나, 우리가 살고 있는 지구가 둥글다는 것을 인류가 깨닫기 시작한 것은 대략 2,000년 전부터이다. 또한 오늘날에는 지구가 완벽한 구 가 아니라는 것도 밝혀졌다. 극반지름은 적도반지름보다 작지만 그 차이는 약 40 km로 크지 않다. 이러한 차이가 나타나는 이유는 지구의 자전에 의해 자전축의 수직방향으로 작용하는 원심력 때문이다.
지구타원체는 실제 지구의 모습과 가장 비슷하게 나타낸 수학적 모델이다. 지금까지 지 구타원체에 대한 다양한 모델이 제시되어 왔으며, 주요 모델은 <표 1?1>과 같다. 그러나 지 구 내부가 불균일하기 때문에 지역에 따라 가장 적합한 지구타원체는 다르다. 실제로〈표 1?1〉에 제시된 바와 같이 나라별로 각 나라에 가장 적합하다고 판단되는 지구타원체를 사용하고 있다.


<표 1-1> 주요 지구타원체

년도
타원체
a(m)
b(m)
1/f
사용국가
1830
Airy
6,377,563.40
6,356,256.90
299.3249
영국
1830
Everest
6,377,276.35
6,356,075.42
300.8017
인도, 미얀마, 파키스탄, 대만
1841
Bessel
6,377,397.16
6,356,078.96
299.1528
한국, 일본, 중국, 독일, 칠레
1866
Clark
6,378,206.40
6,356,583.80
294.9787
북미,
1880
Clark
6,378,249.15
6,356,514.87
293.465
아프리카 대부분, 프랑스
1909
Hayford
6,378,388.00
6,356,911.95
297
북아프리카, 유럽
1948
Krasovsky
6,378,245.00
6,356,863.02
298.3
러시아
1967
GRS67
6,378,160.00
6,356,774.72
298.25
남미, 호주
1972
WGS72
6,378,135.00
6,356,750.50
298.26
미국
1979
GRS80
6,378,137.00
6,356,752.31
298.2572
가장 최근 국제 공인 타원체
1984
WGS84
6,378,137.00
6,356,752.31
298.2572
GPS 좌표체제


지구타원체의 장반경을 a, 단반경을 b라고 하였을 때 편평도 f는 다음과 같다.

<수식>


또한 이심률 e는 다음과 같이 정의된다.

<수식>


위 두 식에 의해 편평도와 이심률의 관계는 다음 과 같이 나타낼 수 있다.

<수식>

그림 1-1 타원의 요소


따라서 타원체의 크기와 형상은 위의 a, b, (f 또는 e)의 요소 중 2개를 알면 결정되는데, 일반적으로 지구타원체는 그림 1-1에서 볼 수 있듯이 장반경 a와 편평률 f로 표현된다.
가장 최근에 알려진 GPS 좌표 체제에 의해 결정된 WGS84 지구타원체의 모델에 의하면 a=6,378,137m, f= , e=0.082이다.

\subsubsection{지오이드}\index{지오이드}


\subsection{지구의 크기}\index{}
\subsubsection{에라토스테네스의 방법}\index{}
\subsubsection{고도에 따른 지평선의 변화}\index{}

\subsection{지구의 질량과 밀도}\index{}
\subsubsection{지구의 질량}\index{}


\section{지구의 내부 구조}\index{}
\subsection{지구의 층상 구조}\index{}
\subsubsection{성분에 의한 층상 구조}\index{}
\subsubsection{물리적 성질에 의한 층상 구조}\index{}
	
\subsection{지진파}\index{}
\subsubsection{탄성반발설}\index{}
\subsubsection{지진파의 종류와 특성}\index{}

\subsection{지구 내부의 물리적 특성}\index{}
\subsubsection{지진파를 이용한 지구 내부의 연구 29 }\index{}
\subsubsection{지구 내부의 밀도, 중력 및 압력}\index{}


\section{지구의 역장}\index{}
\subsection{중력장}\index{}
\subsubsection{지구타원체에서의 중력의 변화}\index{}
\subsubsection{회전하는 지구에서 중력의 변화}\index{}
\subsubsection{지구의 중력 포텐셜}\index{}

\subsection{중력이상}\index{}
\subsubsection{프리에어 중력이상}\index{}
\subsubsection{부게중력이상}\index{}
\subsection{지각평형설}\index{}
\subsubsection{지각평형설의 대두}\index{}
\subsubsection{지각평형모델}\index{}

\subsection{자기장}\index{}
\subsubsection{쌍극자 모형}\index{}
\subsubsection{지구 자기장의 요소}\index{}
\subsubsection{자기장의 변화}\index{}